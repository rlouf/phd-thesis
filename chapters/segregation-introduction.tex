\chapter{What segregation is not}
\label{chap:segregation_introduction}

\begin{flushright}{\slshape    
The limits of my language\\
Mean the limits of my world.} \\ \medskip
--- Ludwig Wittgenstein~\cite{Wittgenstein:1998}
\end{flushright}


\bigskip


% Defining average and variance
\newcommand{\E}{\mathrm{E}}
\newcommand{\Var}{\mathrm{Var}}

\section{Studying segregation}
\label{sec:studying_segregation}


We cannot judge the spatial repartition of people. There is no criterion of
`good' or `bad' in the way people arrange themselves, no moral values attached
to any spatial pattern. It is the \emph{processes} that lead to such patterns,
the intentions behind people's decisions that make segregation condemnable. It
is the \emph{consequences} of segregation that may make undesirable, something
worth fighting against. 

% processes
% Link to models.
% We will not be interested into models, because empirical evidence is too
% shallow
The processes behind segregation are unclear. Is it
because of an attraction to similar individuals, or a repulsion towards
individuals that are different from us? Or maybe is it just simple economic
mechanisms that are self-reinforced over time? If anything, the relatively
recent process of gentrification shows that the answer is not as simple as it
may have seemed at first: wealthy households moving to run-down nighbourhoods
tends to show that. Although simplistic, Schelling's model of segregation shows
that segregation does not necessarily stem from evil intentions.\\
Although it a priori seems that
the former aspect are a topic for qualitative research, being able to follow
households and study their pattern of relocation would also help in studying
the processes behind segregation. Schelling's cellular automata
model~\cite{Schelling:1971}, although intellectually stimulating, is very
limited in terms of predictions. A study of the dynamics of households
(spatially but also in terms of the evolution of their characteristics) should
deliver some precious insights about how individuals interact with their
surroundings.

About models: \cite{Gauvin:2013, Brueckner:1999, Glaeser:2008}

In this thesis, we will not focus on models and mechanisms of residential
segregation. It is our belief that the available empirical characterisation of
segregation is not sufficient. We acknowledge the necessity to
understand the mechanisms that lead to the spatial differentiation of
individuals from different income categories. But we also note the urgent need
to properly characterize first the spatial patterns of segregation.\\

% consequences
% need for a local information
As a matter of fact, social residential segregation has terrible consequences.
As shown in~\cite{Massey:1993}, residential segregation is the cause of major
economic disadvantages that affect the least affluent segments of the
population, through the isolation from social networks, or the presence of poor
public service in the least affluence zones. Worse, it has been shown that
increased levels of segregation in urban areas is associated with a higher
mortality burden~\cite{Lobmayer:2002}. For all these reasons, there is a
somewhat urgent need to measure the extent of segregation, especially its local
component. Most authors systematically design a single index of segregation for
territories that can be very large, up to thousands of square
kilometers~\cite{Apparicio:2000}. In order to mitigate segregation, a more
local, spatial information is however needed: local authorities need to locate
where the poorest and richest concentrate if they want to design efficient
policies to curb, or compensate for, the existing segregation. In other words,
we need to provide a clear {\it spatial} information on the pattern of
segregation. 



\section{Think first, measure later}
\label{sec:introduction}

% the curse of familiarity 
As stated many times, and at different periods in the sociology
literature~\cite{Duncan:1955,James:1982,Massey:1988,Reardon:2002}, the study of
segregation is cursed by its intuitive appeal. Pretty much everyone has heard of
segregation, and has an opinion about it. This familiarity with the concept
favours what Duncan and Duncan~\cite{Duncan:1955} called `naive operationalism':
the tendency to force a sociological interpretation on measures that are at odds
with the conceptual understanding of segregation. In their own words

\begin{quote}
    [Segregation] is a concept rich in theoretical suggestiveness and of
    unquestionable heuristic value. Clearly we would not wish to sacrifice the
    capital of theoretization and observation already invested in the concept.
    Yet this is what is involved in the solution offered by naive
    operationalism, in more or less arbitrary matching some convenient numerical
    procedure with the verbal concept of segregation... (Duncan and Duncan,
    1955~\cite{Duncan:1955})
\end{quote}

For all its intuitive appeal, segregation is however an intricate, compound
notion whose complexity only reveals itself through careful study. However
tempting it is to start writing measures of segregation, on need first to stop
and think about the meaning of this notion. We need to \emph{think} segregation
being able to provide \emph{useful} measures of segregation.


\section{The dimensions of segregation}
\label{sec:the_dimensions_of_segregation}

Luckily for us, a lot of thinking has already been done in the sociology and
geography literature. Previous studies distinguishes between several conceptually
different dimensions.  Massey~\cite{Massey:1988} first proposed a list of $5$
dimensions (and related existing measures), which was recently reduced to $4$ by
Reardon~\cite{Reardon:2004}. 

\subsection{Exposure}
\label{sub:exposure}

(i) {\it exposure} which measures the extent to
which different populations share the same residential areas; 

\subsection{Evenness (clustering)}
\label{sub:evenness_clustering_}

(ii) the {\it
evenness} (and {\it clustering}) to which extent populations are evenly spread
in the metropolitan area; (iii) 

\subsection{Concentration}
\label{sub:concentration}

{\it concentration} to which extent populations
concentrate in the areal units they occupy; and (iv) 

\subsection{Centralization}
\label{sub:centralization}

{\it centralization} to
which extent populations concentrate in the center of the city.\\


\section{The unsegregated city}
\label{sec:null_model_the_unsegregated_city}

The fundamental issue with this picture lies in the lack of a general theoretical framework
in which all existing measures can be interpreted.  Instead, we have a
patchwork of seemingly unrelated measures that are labelled with
either of the aforementionned dimensions. Although segregation can
indeed manifests itself in different ways, it is relatively
straightforward to define what is \emph{not} segregation: a spatial
distribution of different categories that is undistinguishable from a
uniform random situation (with the same percentages of different
categories)~\cite{Jahn:1947}. Therefore, we can define segregation in the following way 

\begin{quote}
Segregation is any pattern in the spatial distribution of categories that significantly 
deviates from a uniform random distribution. 
\end{quote} 

Our definition is perfectly agnostic with regards of the other features of the
density pattern. It is also not concerned with the overall inequality levels. 


In the context of residential segregation, a natural null model is the
\emph{unsegregated city}, where all households are distributed at
random in the city with the further constraints that

\begin{itemize}
    \item The total number $N_\alpha$ of people belonging to a category
	    $\alpha$ is fixed and equal to that found in the data;
    \item The total number $n(t)$ of households living in the areal unit $t$ is
	    fixed and equal to that found in the data.
\end{itemize}

which also fix the total number of individuals $N$ in the city. The problem of
finding the numbers $\left( n_\alpha(1), \dots, n_\alpha(T) \right)$ of
individuals belonging to a certain category $\alpha$ in the $T$ areal units of
an unsegregated city is reminiscent of the traditionnal occupancy problem in
combinatorics~\cite{Feller:1950}. Their distribution is given by the multinomial
distribution $f \left( n_\alpha(1), \dots, n_\alpha(T) \right)$, and the number
of people of category $\alpha$ in the areal unit $t$ by a binomial distribution.
Therefore, in an unsegregated city, we have

\begin{align}
    \begin{split}
	\E \left[ n_\alpha(t) \right] &= N_\alpha\,\frac{n(t)}{N} \\
	\Var \left[ n_\alpha(t) \right] &= N_\alpha\,\frac{n(t)}{N} \left( 1 - \frac{n(t)}{N}  \right) 
    \end{split}
\end{align}

where $N$ is the total number of households in the city. In metropolitan areas
$N_\alpha$ is larged compared to $1$, and the distribution of the $n_\alpha(t)$
can be approximated by a Gaussian with the same mean and variance.

The different dimensions of~\cite{Massey:1988,Reardon:2004} correspond then to
particular aspects of how a multi-dimensional pattern can deviate from its
randomized counterpart. The measures we propose here are all rooted in this
general definition of segregation.
Another advantage of the above definition of segregation with a null model is
that we are able to evaluate whether the pattern deviates significantly from the
random configuration.
Most studies exploring the question of spatial segregation define measures
before comparing their value for different cities. Knowing that two quantities
are different is however not enough: we also have to know whether this
difference is {\em significant}. In order to assess the significance of a
result, we have to compare it to what is obtained for a reasonable null model.\\

\bigskip

In this chapter, we have discussed some of the improvements that could be
brought to the existing measures in the literature. In particular, we have
emphasized the need for a \emph{local} knowledge of the patterns of segregation.
We have also laid the theoretical foundation upon which we are going to design
new segregation measures.
