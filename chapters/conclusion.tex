% !TEX root = ../thesis-example.tex
%
\chapter{Conclusion}
\label{sec:conclusion}

\begin{flushright}{\slshape    
If people never did silly things\\
nothing intelligent would ever get done.} \\ \medskip
--- Ludwig Wittgenstein~\cite{Luckhardt:1979}
\end{flushright}

\bigskip

\section{What the past 3 years have brought}
\label{sec:what_the_past_3_years_have_brought}


What I find striking is the propention of the different communities to ignore
one another. Although there is a lot to say about physicists being ignorant of
most of the literature in social sciences, there is also a lot to say about
social sciences applying tools as black boxes, without any reflexion on the
meaning of the formula that is being applied. Despite an important literature on
the topic, we can still see exponents of power-law distribution estimated using
the Least-Squares method. A lot of the litterature on complex network is also
applied carelessly, a big victim being the algorithms of community detection (no
reflection on the meaning of communities, for instance). It is a real shame,
because we need people with quantitative skills to manipulate formulas and
create new measures, but we also need qualitative skills to manipulate concept
and interpret the significance of the results. While we cannot expect a
physicist or a computer scientitist to know all the literature in a field that
is not his own, we cannot expect social scientists to have strong quantitative
skills. The real solution resides in the collaboration.\\

It is ignorance, not knowledge, that fuels science.

\section{The thesis I wish I wrote}
\label{sec:limitations}

The temptation is great, looking back on $3$ years of work with a more
experienced eye, to understate the contributions of this thesis and their
potential applications. The thesis that you have been reading is of course very
different from the thesis I would have liked to write. Frustrating realisation
that good research takes time. It takes time for the literature to sink in, it
takes time to understand the limits of your contribution, it takes time to get
at the bottom of a field. Not that I would do anything differently---I
couldn't--- but I now start to realise the work that is yet to be accomplished.
And that is only for what I am aware should be done, new problems and questions
are ready to pop out of nowhere at any time.\\

So what would have I write about---or at least try to---if I had to start my
thesis all over again? This is another way of saying: what are the next steps?
How does the research presented here fit in this picture?\\

I would start being more careful with the concepts that are being used. Starting
with the basics, with the single noun that was most often printed in these
pages: Cities. Despite it being our object study, it seems that the very
definition of this object is something obscure, somewhat hidden in the
literature. At least, it is something that is not really talked about in the
literature. Yet, if we want to exhibit robust empirical results, we need to
start worrying about the definition of the system we are studying. We need to
know \emph{what} cities we are talking about.\\


Once the boundaries are defined, we can start studying the way objects are
scattered within them. By objects, I mean buildings, roads, and first and
foremost people. The way we traditionally study the repartition of objects in
space is through the study of densities. An interesting thing to study would for
instance be the population density profile, possibly at different times of the
day. But density profiles are too complicated to comprehend for our brains,
especially when cities get large. Spatial statistics attemps to solve this
problem by providing simple measures, that extract a single number from
distributions. A single number is however too simple to be able to describe
accurately complex density profiles. What we need is a meso-scale
representation, somewhere between the micro-scale picture (the density profile
itself) and the macro-scale picture (a single number to summarize the density
profile). A requirement for this method is to make the definition of centers (or
hotspots) natural, because centers are a meso-scale representation of density
profiles already. This should make the notion of a center defined from first
principle, which would then allow to discuss the \emph{absolute} number of
centers in the city. Until now, because we do not clearly understand what the
meaning of the centers we obtain with the different existing methods, we are
only able to get numbers that are useful for inter-urban comparisons. The
spatial contiguity is not even taken into account!  This is where the model on
the polycentric transition of cities, presented in
Chapter~\ref{chap_monocentric} fits. Although I do not expect the scaling of the
number of centers with city size to change (but who knows), having a more
accurate description of the poly- or monocentric structure of population
distributions at different times of the day should help put more constrains on
existing models, and give hints for improvement of the existing models.\\


Once one is able to provide an accurate description of densit profiles, the
possibilities start to diverge. An obvious worry, when one has a picture of the
city's population at different times of the day, is the way these profile
transform one into another. This is linked to commuting---but not only,
commuting representing only $20\%$ of total travels in the
US~\cite{FHWA-PL-11-022}--- and the study of congestion of networks. We can
first wonder the link between the urban form (typically the residential and
employment densities) and mobility patterns. This is tackled in the excess
commuting literature (a few references here). There fit both the first and
second part of this thesis.\\ A futher worry linked to commuting is that of
congestion: understanding how congestion are formed, how they propagate and
devise strategies to mitigate them, either by influencing the transportation
infrastructure, the spatial repartition of residences and employment, or the
behaviour of people themselves.  This is far from being a recent worry, and
there already is an impressively vast literature on the topic [cite review].
Yet, there is room for new approaches that leverage the knowledge we have about
network and phase transition in physics. A first step in this direction has been
made by the authors of~\cite{Daqing:2015}, but there is surely more to be
understood and discovered.  Modeling congestion also implies understanding the
individual behaviour of people when they are moving from a point to another in
cities. Altough most research nowadays assume that people choose the shortest
(time or distance) path, GPS data now provide overwhelming evidence that this is
not the case [Manley, etc]. So, while there is a clear need to understand the
meso-scopic picture (how congestion spread), there is also is a critical need to
understand the microscopic picture (how people behave).\\

So far we have talked about the movement induced by the spatial mismatch between
residential areas and activity areas. One might also want to study the
characteristics of the spatial repartition of people. Inhabitants of cities are
not just a combination of a latitude and a longitude, a point on a map. Like you
and me, they are characterised by different qualities, some of which are
measurable: say their income, their education level, their ethnicity, etc. A
natural question, that has interested sociologist and geographers, is to wonder
whether people's residence is independent of these characteristics, or whether
these characteristics have an influence on the spatial repartition of
individuals. In other words, the study of residential segregation. A
contribution of this thesis is to provide a rigorous method to study the
patterns of segregation in the presence of multiple categories, and to identify
`neighbourhoods', that is regions in the city where individual belonging to a
particular category. I hope that, along with the improvement of the
characterisation of density profiles, we will be able to describe more
accurately the spatial patterns of segregation, which should hopefully lead to a
better understanding of this phenomenon, and better models.\\

The list is still long, and gets more and more speculative as we deviate from
the questions that are directly entailed by the work I have presented here.


\section{Future work}
\label{sec:future_work}

I do believe there is a cruel lack of serious empirical work in the field. This
should be quickly solved, thanks to the information technologies that are now
available. But there is a bigger problem looming over our heads. The
uncomfortable fact that our fundamental object, the city, is an ill-defined
object. And that most empirical studies possibly rely on definitions of a city
that are not suited to the study they undertake. 
This lack of serious definition compromises the comparison between cities of
different countries (as I saw in the fingerprint paper), or different points in
time. I am, of course, not the first person to acknowledge this empirical
difficulty. In fact, it has been a long-time worry of geographers who have been
trying to produce harmonised database for a long time (cite
Bretagnolle, Pumain, any reference on harmonised database). Yet, we still lack
of an unambiguous, theoretically grounded definition of what a city is. And this
is problematic, since statistical institutes' results are based on what is
believed to be the best definition of the city at a time, which influences the
results, etc.
The problem is that it is unlikely there is only one way to define the object
city, but rather different definitions that depend on what we are studying.
People are usually confused by the existence of different definitions.


