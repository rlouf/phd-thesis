% Abstract

\pdfbookmark[1]{Abstract}{Abstract} % Bookmark name visible in a PDF viewer

\begingroup
\let\clearpage\relax
\let\cleardoublepage\relax
\let\cleardoublepage\relax

\chapter*{Abstract} % Abstract name


The study of cities is not new, and traditionally pertains to the fields of
Geography, Economics, Sociology, Urbanism, etc.  The last decade has witnessed a
renewed interest in these systems, attracting scholars from many disciplines,
including physicists.

The amount of data that is being gathered about cities is increasing in size and
specificity. 

We are interested in various aspects of the intra-urban dynamics. How can we
understand the transition of cities from a monocentric to a polycentric
organisation? How can we understand the scaling exponents related to mobility?
Finally, how can we provide a measure of segregation

% Monocentric to polycentric transition
We present a stochastic, out-of-equilibrium model of city growth that describes
the structure of the mobility pattern of individuals. The model explains the
appearance of secondary subcenters as an effect of traffic congestion. We are
also able to predict the sublinear increase of the number of centers with
population size, a prediction that is verified on American and Spanish data. 
% Scaling
In the framework of this model, we are further able to give a prediction for the
scaling exponent of the total distance commuted daily, the total
length of the road network, the total delay due to congestion, the quantity of $CO_2$
emitted, and the surface area with the population size of cities. Predictions
which agree with the data gathered for U.S. cities.
% Segregation

%In a first part, we outline the specificity and importance of studying cities,
%introduce briefly the state-of-the-art in urban analysis before discussing our
%methodological stance.

%We then proceed in the following part to the study of polycentricity of cities.
%We prove empirically the existence of a polycentric transition of cities.
%Prediction that agrees with the available date.

%In the second part, we tackle scaling relationships, which are the object of a
%renewed interest. Fist, we provide a literatu


We try to convey that the complexity of cities is -- almost paradoxically --
better comprehended through simple approaches. The identification and definition
of the relevant systems at play, as well as the hierarchisation of processes are
fundamental if we want to build a corpus of knowledge on which future research
can be based.  Looking for structure (stylized facts) in data, trying to isolate
the most important processes, building simple models and only keeping those
which agree with data, constitute a universal method that is relevant to the
study of any system.


\endgroup			

\vfill
